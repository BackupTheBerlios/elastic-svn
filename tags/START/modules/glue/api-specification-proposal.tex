% -*- Latex -*-
% Copyright (C) 2002 Marco Pantaleoni. All rights reserved.
%
% $Id$

\documentclass[10pt,twocolumn]{article}

%% Use the following two lines for PDF (pdfLaTeX)
%% and commend 'cmbright' below
%\usepackage[T1]{fontenc}
%\usepackage{times}

% Use this for normal (not pdfLaTeX) output
\usepackage{cmbright}

\pagestyle{empty}

\begin{document}

\title{A proposal for C API specification for automated binding
  generation for interpreted languages}

\author{Marco Pantaleoni\\ {\small panta@elasticworld.org}}

\date{30 May 2002}

\maketitle

\thispagestyle{plain}

\section{Introduction}

Usually interpreted languages can be extended by adding new functions
to the set they provide on their own. In this document we'll consider
those languages that can be extended by writing C modules that can be
linked into the interpreter either statically or dynamically.
There is a wealth of C libraries whose functionalities would be useful
to these interpreted languages, but that have not yet been integrated
due to the time-consuming and boring process of writing bindings by
hand.\par
Our objective is to develop a formalism to specify C APIs in such a
way that bindings to interpreted languages can be generated easily.
With our formalism, we must be able to write a concise description of
data structures and function prototypes. Then processing the
description with a binding generator we'll obtain a module
implementing the described types and functions in our language of
choice.

\section{Objectives}

Our formalism has to satisfy some requirements in order to be useful:

\begin{itemize}
  \item writing specifications must be easier than writing bindings
    for the majority of interpreted languages;

  \item the formalism must be powerful enough to allow capturing all
    the semantic aspects of a common C library;

  \item the formalism must take into account typical interpreted
    language needs and mechanisms, like automatic memory management
    (\textsl{garbage collection}), high level types, \dots

  \item it has to be easily machine-readable
\end{itemize}

It is important to note the relevance of the second point in the above
list: automated header file processing cannot satisfy this point,
since C header files can't capture the complete semantic of a C
library.\footnote{Think for example to pointer arguments to functions:
  is the argument somewhat to be considered as input or output, or
  both? It is impossible to tell from the simple examination of a
  function prototype.}

\section{Lisp-like syntax}

Lisp syntax seems perfect for such a formalism: it's exceptionally
simple, it is very easy to parse and its list structure makes it
ideal for the specification of complex data structures and function
prototypes.\par
We won't describe in detail the Lisp syntax here, we'll just limit to
recall that the basic syntactical elements in Lisp are atoms, the most
primitive entities, and lists\footnote{LISP stands for \textsl{LISt
    Processing}}, which are ordered sequences of elements, where each
element can be an atom or a list.\par
The atoms we'll use will be symbolic names and numerical values. In
our formal notation, atoms will be strings made of alphanumeric
characters plus \texttt{.} (dot), \texttt{-} (minus), \texttt{\_}
(underscore) and \texttt{"} (double quote). We'll use atoms as
symbolic names, numerical values and strings.\par
Examples of valid atoms are:

\texttt{
\begin{itemize}
\item c-int
\item c-long
\item strlen
\item 10
\item "time.h"
\item struct
\end{itemize}
}

Lists are specified as a possibly empty sequence of space-separated
elements, enclosed in parenthesis. Examples:

\texttt{
\begin{itemize}
\item (a b c)
\item ()
\item (struct point ((c-float x) (c-float y)))
\end{itemize}
}

\section{C API specification}

In our formal specification we distinguish between type specification,
function specification and directives.

\subsection{Type specification: atomic and derived types}

Types can be either \textsl{atomic} or \textsl{derived}. Atomic types
are those that in the interpreted language are considered primitve
(integers, floating point, strings, ...), corresponding directly to C
basic types (like \texttt{int}, \texttt{float}, \dots)\footnote{we
  considered the string an atomic type, even if its C counter-part is
  a complex type (a pointer to char: \texttt{char *}), because
  high-level interpreted languages usually provide a primitive string
  type. The atomic-ness property is to be considered on the
  interpreted language side, not at C level.}, or corresponding to
\texttt{typedef}s of basic C types (like, say, \texttt{time\_t} in
\texttt{<file.h>}).\par

\subsubsection{Atomic types}

The formalism has provision for some existing basic types, while
others can be added using directives described below.
Pre-existing atomic types in our formalism are:

\begin{itemize}
\item \texttt{int}, \texttt{sint}, \texttt{uint} - corresponding
  respectively to C native \texttt{int}, \texttt{signed int},
  \texttt{unsigned int}
\item \texttt{long}, \texttt{slong}, \texttt{ulong} - corresponding
  respectively to C native \texttt{long}, \texttt{signed long},
  \texttt{unsigned long}
\item \texttt{short}, \texttt{sshort}, \texttt{ushort} - corresponding
  respectively to C native \texttt{short}, \texttt{signed short},
  \texttt{unsigned short}
\item \texttt{char}, \texttt{schar}, \texttt{uchar} - corresponding
  respectively to C native \texttt{char}, \texttt{signed char},
  \texttt{unsigned char}
\item \texttt{int8}, \texttt{uint8} - corresponding
  respectively to C native 8 bit signed and unsigned integers
\item \texttt{int16}, \texttt{uint16} - corresponding
  respectively to C native 16 bit signed and unsigned integers
\item \texttt{int32}, \texttt{uint32} - corresponding
  respectively to C native 32 bit signed and unsigned integers

\item \texttt{string-in}
\item \texttt{string-out}
\item \texttt{string-inout}
\item \texttt{string-m-out}
\end{itemize}

New atomic types corresponding to a C \texttt{typedef} can be added
using the \texttt{c-atomic-type} directive:

\begin{quotation}
  \texttt{(c-atomic-type "\textsl{ctype}")}
\end{quotation}

For example, to add the \texttt{time\_t} type defined in
\texttt{<time.h>}:

\begin{quotation}
  \texttt{(c-atomic-type "time\_t")}
\end{quotation}

It is possible to add a new atomic type corresponding to a C
enumeration or to a set of \texttt{\#define}s, using the \texttt{enum}
form:

\begin{quotation}
  \texttt{(c-enum-type \textsl{enum-name} \textsl{"c-type-name$_{opt}$"} ((\textsl{symbolic-name$_1$} \textsl{C-value$_1$})\dots ))}
\end{quotation}

(the \textsl{c-type-name} specifies the C name of the enumeration. It
is mandatory for enumerations, unused otherwise).\par

For example, if we have the C fragment:

\begin{verbatim}
    typedef enum { color_red, color_green, color_blue } color_t;
\end{verbatim}

we would write:

\begin{verbatim}
    (c-enum-type color "color_t"
      ((red color_red)
       (green color_green)
       (blue color_blue)))
\end{verbatim}

If instead we have:

\begin{verbatim}
    #define COLOR_RED   1
    #define COLOR_GREEN 2
    #define COLOR_BLUE  3
\end{verbatim}

we could use:

\begin{verbatim}
    (c-enum-type color
      ((red COLOR_RED)
       (green COLOR_GREEN)
       (blue COLOR_BLUE)))
\end{verbatim}

\subsubsection{Derived types}

\subsubsection{Type attributes}

const, volatile, \dots

\subsection{Function prototype specification}

\subsection{Directives}

\end{document}
