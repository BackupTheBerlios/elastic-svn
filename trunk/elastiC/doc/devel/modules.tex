% -*- Latex -*-
% $Id$

\documentclass[makeidx,10pt,titlepage]{article}
\usepackage{supertabular}
\usepackage{color}
\usepackage{algorithm}

\newif\ifpdf
\ifx\pdfoutput\undefined
  \pdffalse
\else
  \pdfoutput=1
  \pdftrue
\fi


\ifpdf
  %% Use the following two lines for PDF (pdfLaTeX)
  %% and commend 'cmbright' below
  \usepackage[T1]{fontenc}
  \usepackage{times}
\else
  % Use this for normal (not pdfLaTeX) output
  \usepackage{cmbright}
\fi

\begin{document}
\title{elastiC Module system}
\author{
Marco Pantaleoni\\
{\sl panta@elasticworld.org}}
\date{June 14, 2002}
\maketitle

\begin{abstract}
We want to present the specification and rationale behind elastiC
module system. elastiC modules provide a flexible mechanism to
encapsulate a common set of functionalities.
\end{abstract}

\section{Introduction}

elastiC modules, or packages, provide a mean of collecting together
variables (and hence functions and classes) into separate namespaces,
thus preventing accidental name clashes.
An elastiC program is itself necessarily a module.

\section{Usage}
 A module can access variables declared in a different module by 1)
 {\sl importing} the desired package and 2) accessing
the name in the imported module by {\sl qualifying} the variable with
the module name\footnote{we'll see that it is also possible to import
  some simbols into the current namespace, thus avoiding the need to
  fully qualify their names.}.
Modules can be nested to increase the flexibility in decomposing
complex projects into manageable pieces.

\end{document}
